\documentclass{article}%
\usepackage[T1]{fontenc}%
\usepackage[utf8]{inputenc}%
\usepackage{lmodern}%
\usepackage{textcomp}%
\usepackage{lastpage}%
\usepackage{amsmath}%
%
%
%
\begin{document}%
\normalsize%
\[%
%
\]%
\section{Reduccion y factorizacion de matriz}%
\label{sec:Reduccionyfactorizaciondematriz}%
Vamos a hacer reduccion de Gauss tomando en cuenta la siguiente estructura:%
\[%
\newline%
Fila{[}x{]}{-}(k)Fila{[}y{]}%
\]%
Siendo:%
\[%
\newline%
Fila{[}x{]}%
\]%
La fila la cual queremos reducir%
\[%
\newline%
Fila{[}y{]}%
\]%
La fila a la cual pertenece el pivot\newline%
%
\[%
\newline%
(k)%
\]%
El numero a ser multiplicado para permitir la reduccion\newline%
%
\subsection{Matriz Ingresada: }%
\label{subsec:MatrizIngresada}%
\[%
\left[\begin{matrix}3 & 1 & 0\\-12 & 0 & 2\\3 & 4 & 5\end{matrix}\right]%
\]

%
\subsection{Procedimiento}%
\label{subsec:Procedimiento}%
\[%
Fila{[}2{]}{-}({-}4)*Fila{[}1{]}%
\]%
\[%
\left[\begin{matrix}1 & 0 & 0\\-4 & 1 & 0\\0 & 0 & 1\end{matrix}\right] \left[\begin{matrix}3 & 1 & 0\\0 & 4 & 2\\3 & 4 & 5\end{matrix}\right]%
\]%
\[%
Fila{[}3{]}{-}(1)*Fila{[}1{]}%
\]%
\[%
\left[\begin{matrix}1 & 0 & 0\\-4 & 1 & 0\\1 & 0 & 1\end{matrix}\right] \left[\begin{matrix}3 & 1 & 0\\0 & 4 & 2\\0 & 3 & 5\end{matrix}\right]%
\]%
\[%
Fila{[}3{]}{-}(3/4)*Fila{[}2{]}%
\]%
\[%
\left[\begin{matrix}1 & 0 & 0\\-4 & 1 & 0\\1 & \frac{3}{4} & 1\end{matrix}\right] \left[\begin{matrix}3 & 1 & 0\\0 & 4 & 2\\0 & 0 & \frac{7}{2}\end{matrix}\right]%
\]

%
\subsection{Resultado}%
\label{subsec:Resultado}%
Factorizacion matricial:%
\[%
\newline%
 A     = L       U%
\]%
\[%
\left[\begin{matrix}3 & 1 & 0\\-12 & 0 & 2\\3 & 4 & 5\end{matrix}\right] = \left[\begin{matrix}1 & 0 & 0\\-4 & 1 & 0\\1 & \frac{3}{4} & 1\end{matrix}\right] \left[\begin{matrix}3 & 1 & 0\\0 & 4 & 2\\0 & 0 & \frac{7}{2}\end{matrix}\right]%
\]

%
\end{document}